\documentclass[14pt, a4paper]{article}


	\begin{document}
		
				
		\title{ MAKERERE UNIVERSITY \\COLLEGE OF COMPUTING AND INFORMATION SCIENCES\\ Concept document of Automated Analysis of Nash Equilibria in Iterated Boolean Games }

		
		\author{\begin{tabular}{ |p{6cm}|p{5cm}|p{4cm}|  }
\hline
\multicolumn{3}{|c|}{GROUP 117} \\
\hline
NAME & REGISTRATION NO. &STUDENT NO. \\
\hline
Sendikaddiwa Marvin & 15/U/1154 &215000166 \\
\hline
Matovu Joseph & 15/U/7462/PS &215011917 \\
\hline
Majanga Joseph&15/U/7319/PS& 215013152\\
\hline
Kahumuza Benon  &14/U/6794/PS &214013697  \\
\hline
\end{tabular}}

		\date {\today}

		\maketitle

		\tableofcontents

			\section{Introduction}
Nash equilibrium is a solution concept of a non-cooperative game involving two or more players in which each player is assumed to know the equilibrium strategies of the other players, and no player has anything to gain by changing only his or her own strategy. If each player has chosen a strategy and no player can benefit by changing strategies while the other players keep theirs unchanged, then the current set of strategy choices and the corresponding payoffs constitutes a Nash equilibrium.

Nash equilibrium is also a fundamental concept in the theory of games and the most widely used method of predicting the outcome of a strategic interaction in the social sciences. Agame consists of three elements : a set of players, set of actions available to each player and a payoff function for each player. The payoff fuctions represent each player's preferences over action profiles, where an action profile is simply a list of actions, one for each player. 

				

			\section{Keywords}
			
				LTL - Linear Temporal Logic, MCMAS - Model Checker for Multi-Agent Systems, ISPL- Interpreted Systems Programming Language.

				

			\section{Background to the problem}
 Nash equilibrium concept  makes misleading predictions (or fails to make a unique prediction) in certain circumstances.  many related solution concepts have been proposed  (also called 'refinements' of Nash equilibria) designed to overcome perceived flaws in the Nash concept. One particularly important issue is that some Nash equilibria may be based on threats that are not 'credible'. In 1965 Reinhard Selten proposed subgame perfect equilibrium as a refinement that eliminates equilibria which depend on non-credible threats. Other extensions of the Nash equilibrium concept have addressed what happens if a game is repeated, or what happens if a game is played in the absence of complete information. However, subsequent refinements and extensions of the Nash equilibrium concept share the main insight on which Nash's concept rests: all equilibrium concepts analyze what choices will be made when each player takes into account the decision-making of others.

Now consider the game that involves a repetition of the prisoner’s dilemma for n periods, where n is commonly known to the two players. A pure strategy in thisrepeated game is a plan that prescribes which action is to be taken at each stage, contingent on every possible history
of the game to that point. Clearly the set of pure strategies is very large. Nevertheless, all Nash equilibria of this finitely repeated game involve defection at every stage. When the number of stages n is large, equilibrium payoffs lie far below the payoffs that could have been
attained under mutual cooperation. It has sometimes been argued that the Nash prediction in the finitely repeated prisoner’s dilemma (and in many other environments) is counterintuitive and at odds with experimental evidence. However, experimental tests
of the equilibrium hypothesis are typically conducted with monetary payoffs, which need not reflect the preferences of subjects over action profiles. In other words, individual preferences over the distribution of monetary payoffs may not be exclusively self-interested.
Furthermore, the equilibrium prediction relies on the hypothesis that these preferences are commonly known to all subjects, which is also unlikely to hold in practice.

				
			\section{problem statement}
			
			
			\section{Aim and objectives}
			
				\subsection{Aim or General Objective}
					To Automate the analysis of Nash equilibria in Boolean iterated games\\
		To check whether Multiplayer games can be solved in practice\\
		To generate an Algorithm that will check the performance of Boolean iterated games.\\
		
				
				\subsection{specific objectives}
						To introduce a novel notion of expressiveness for temporal logics that is based on game theoretic properties of multi-agent systems.\\
		To apply the standard game-theoretic concept of Nash equilibria.\\

			
			\section{Research scope}
			
				The scope of this project is between multiplayer games of only two players thus if the game includes one player of more than two it will be excluded in our research.\\
	We study the problem of computing pure-strategy Nash equilibria in multiplayer concurrent games.\\
	The analysis of Nash equilibria will be concluded with a general approximation other than specifying an accurate formulae.(ie using their expressiveness powers)\\

	In this model, each agent i exercises exclusive control over a subset of Boolean variables, and the game is played over an infinite number of rounds, where at each round each player chooses a valuation for their variables.\\

	Each player is assumed to act strategically, taking into account the goals of other players,
in order to try to bring about computations that will satisfy their goal.\\

			
			\section{Research Significance}
			
				To find out the running times (Analyse) and check whether Nash equilibria can be obtained in multiplayer games and to deduce their complexity.
				To check out whether artificial intelligence algorithms can be implemented in the multiplayer 
games.

			
			\section{References}
R. Alur, T. A. Henzinger, and O. Kupferman. Alternating-time
temporal logic. Journal of the ACM, 49(5):672–713, 2002.\\
C. Baier and J.-P. Katoen. Principles of Model Checking. The
MIT Press, 2008.\\
E. W. Beth. On Padoa’s method in the theory of definition.
Indagationes Mathematicae, 15:330–339, 1953.\\
E. Bonzon, M. Lagasquie, J. Lang, and B. Zanuttini. Boolean
games revisited. In Proceedings of the Seventeenth European
Conference on Artificial Intelligence (ECAI-2006), 2006.\\
				
	

	\end{document}
