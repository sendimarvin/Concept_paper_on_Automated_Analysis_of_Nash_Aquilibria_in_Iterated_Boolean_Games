\documentclass[14pt, a4paper]{article}


	\begin{document}
		
				
		\title{ MAKERERE UNIVERSITY \\COLLEGE OF COMPUTING AND INFORMATION SCIENCES\\ Concept document of Automated Analysis of Nash Equilibria in Iterated Boolean Games }

		
		\author{\begin{tabular}{ |p{6cm}|p{5cm}|p{4cm}|  }
\hline
\multicolumn{3}{|c|}{GROUP 117} \\
\hline
NAME & REGISTRATION NO. &STUDENT NO. \\
\hline
Sendikaddiwa Marvin & 15/U/1154 &215000166 \\
\hline
Matovu Joseph & 15/U/7462/PS &215011917 \\
\hline
Majanga Joseph&15/U/7319/PS& 215013152\\
\hline
Kahumuza Benon  &14/U/6794/PS &214013697  \\
\hline
\end{tabular}}

		\date {\today}

		\maketitle

		\tableofcontents

			\section{Introduction}
Nash equilibrium is a solution concept of a non-cooperative game involving two or more players in which each player is assumed to know the equilibrium strategies of the other players, and no player has anything to gain by changing only his or her own strategy. If each player has chosen a strategy and no player can benefit by changing strategies while the other players keep theirs unchanged, then the current set of strategy choices and the corresponding payoffs constitutes a Nash equilibrium.

Nash equilibrium is also a fundamental concept in the theory of games and the most widely used method of predicting the outcome of a strategic interaction in the social sciences. Agame consists of three elements : a set of players, set of actions available to each player and a payoff function for each player. The payoff fuctions represent each player's preferences over action profiles, where an action profile is simply a list of actions, one for each player. 

				

			\section{Keywords}
			
				LTL - Linear Temporal Logic, MCMAS - Model Checker for Multi-Agent Systems, ISPL- Interpreted Systems Programming Language.

				

			\section{Background to the problem}

				
			\section{problem statement}
			
			
			\section{Aim and objectives}
			
				\subsection{Aim or General Objective}
					To Automate the analysis of Nash equilibria in Boolean iterated games\\
		To check whether Multiplayer games can be solved in practice\\
		To generate an Algorithm that will check the performance of Boolean iterated games.\\
		
				
				\subsection{specific objectives}
						To introduce a novel notion of expressiveness for temporal logics that is based on game theoretic properties of multi-agent systems.\\
		To apply the standard game-theoretic concept of Nash equilibria.\\

			
			\section{Research scope}
			
				The scope of this project is between multiplayer games of only two players thus if the game includes one player of more than two it will be excluded in our research.\\
	We study the problem of computing pure-strategy Nash equilibria in multiplayer concurrent games.\\
	The analysis of Nash equilibria will be concluded with a general approximation other than specifying an accurate formulae.(ie using their expressiveness powers)\\

	In this model, each agent i exercises exclusive control over a subset of Boolean variables, and the game is played over an infinite number of rounds, where at each round each player chooses a valuation for their variables.\\

	Each player is assumed to act strategically, taking into account the goals of other players,
in order to try to bring about computations that will satisfy their goal.\\

			
			\section{Research Significance}
			
				To find out the running times (Analyse) and check whether Nash equilibria can be obtained in multiplayer games and to deduce their complexity.
				To check out whether artificial intelligence algorithms can be implemented in the multiplayer 
games.

			
			\section{References}
R. Alur, T. A. Henzinger, and O. Kupferman. Alternating-time
temporal logic. Journal of the ACM, 49(5):672–713, 2002.\\
C. Baier and J.-P. Katoen. Principles of Model Checking. The
MIT Press, 2008.\\
E. W. Beth. On Padoa’s method in the theory of definition.
Indagationes Mathematicae, 15:330–339, 1953.\\
E. Bonzon, M. Lagasquie, J. Lang, and B. Zanuttini. Boolean
games revisited. In Proceedings of the Seventeenth European
Conference on Artificial Intelligence (ECAI-2006), 2006.\\
				
	

	\end{document}
